% %%%%%%%%%%%%%%%%%% PREAMBLE BEGIN %%%%%%%%%%%%%%%%%%
% A templet for mathematical LaTeX2e documents - partially based on the preambles avaible at http://psi.nbi.dk/groups/psi/wiki/Preamble


%%%%%%%%%%%%%%%%%% ESSENTIAL FORMATTING %%%%%%%%%%%%%%%%%%
\documentclass[a4paper,10pt, landscape]{scrartcl}
% \documentclass[a4paper,11pt,notitlepage]{report}

% If activated, labels will be shown in the margin. Use only when in draft mode.
%\usepackage[notref,notcite]{showkeys}



%%%%%%%%%%%%%%%%%% FOR DANISH DOCUMENTS %%%%%%%%%%%%%%%%%%

% Chapters is given danish names and so on.
% \usepackage[danish]{babel}
\usepackage[english]{babel}

% Makes commas work as decimal-points (fixes the spacing around.
% \usepackage{icomma}

% Sets the indentation for new paragraphs. Per default LaTeX creates some indentation whenever a new paragraph begins. While this might look visually pleasing, it's an english/american convention, and not "correct" danish typestting.
% \parindent=0pt

 % For use on Unix systems (Mac/Linux). Use this or the one below.
\usepackage[utf8]{inputenc}

% For use on Windows systems. Use this or the one above.
% \usepackage[T1]{fontenc}



%%%%%%%%%%%%%%%%%% ESSENTIAL PACKAGES %%%%%%%%%%%%%%%%%%
% Make PDF-files searchable
\usepackage{cmap}


% Typesets the pagenumber of the last page by entering \pageref{Lastpage}
\usepackage{lastpage}

% Overfull hbox under 3pt ignoreres - makes LaTeX a but less fussy, but might also make things look a little bit less pleasing to the eye.
\hfuzz=3pt


% Through tutorial: https://en.wikibooks.org/wiki/LaTeX/Page_Layout
% \usepackage[a4paper]{geometry}

% It's considered obsolete, but it's very explicit and easy to use, which makes it a good alternative for geometry for a quick fix
% Options are: \marginsize{left}{right}{top}{bottom}
\usepackage{anysize}
   \marginsize{1.0cm}{1.5cm}{1.2cm}{1.8cm}


%    ***    Font options    ***
% All fonts include mathematical characters for equations, and support the danish letters æøåÆØÅ.
% Documentation: http://www.tug.dk/FontCatalogue/mathfonts.html
% Remove the comment sign to use a font.

% Palatino
%\usepackage{mathpazo}

% Times
%\usepackage{mathptmx}

% Anonymous Pro for typewriter text  NOTE: Documentation states that \usepackage[T1]{fontenc} and \usepackage{textcomp} must be loaded:
% ftp://ftp.dante.de/tex-archive/fonts/anonymouspro/doc/AnonymousPro.pdf
% \usepackage[ttdefault]{AnonymousPro}


% The Fancyheaders-package makes those fancy header and footer seen in a lot af LaTeX-documents.
\usepackage{fancyhdr}
\pagestyle{fancy}                                                % Initialises layout options for the package
\fancyhead{}                  % Vipes all header-fields
\fancyhead[L]{Question overview}
\fancyhead[R]{Checking for reversed scales}
\headheight 14.5pt            % Sets the height of the header-field.
\fancyfoot{}                  % Vipes all footer-fields
\fancyfoot[C]{Page \thepage{} of \pageref{LastPage}}
\renewcommand{\headrulewidth}{0.2pt}  % Set thickness of the headers line/ruler
\renewcommand{\footrulewidth}{0.0pt}  % Set to 0 (zero) to remove footer rule

% Ensures that first page (front page) have a clean header
\fancypagestyle{firstPageStyle}
{
   \fancyhead{}
   \fancyfoot{}
   \headheight 0pt
   \renewcommand{\headrulewidth}{0.0pt}
   % \fancyfoot[C]{Page \thepage{} of \pageref{LastPage}}
}



%%%%%%%%%%%%%%%%%% ADDITIONAL PACKAGES %%%%%%%%%%%%%%%%%%
% Greatly improves on PDFLaTeX's typesetting
\usepackage{microtype}


% Typeset better tables, with proper vertical spacing, ofr instance.
% Tutorial: http://en.wikibooks.org/wiki/LaTeX/Tables#Professional_tables
% Documentation: http://mirrors.med.harvard.edu/ctan/macros/latex/contrib/booktabs/booktabs.pdf
\usepackage{booktabs}

\makeindex




% Default to the notebook output style




% Inherit from the specified cell style.





\documentclass[a4paper,10pt, landscape]{article}



    \usepackage[T1]{fontenc}
    % Nicer default font (+ math font) than Computer Modern for most use cases
    \usepackage{mathpazo}

    % Basic figure setup, for now with no caption control since it's done
    % automatically by Pandoc (which extracts ![](path) syntax from Markdown).
    \usepackage{graphicx}
    % We will generate all images so they have a width \maxwidth. This means
    % that they will get their normal width if they fit onto the page, but
    % are scaled down if they would overflow the margins.
    \makeatletter
    \def\maxwidth{\ifdim\Gin@nat@width>\linewidth\linewidth
    \else\Gin@nat@width\fi}
    \makeatother
    \let\Oldincludegraphics\includegraphics
    % Set max figure width to be 80% of text width, for now hardcoded.
    \renewcommand{\includegraphics}[1]{\Oldincludegraphics[width=.8\maxwidth]{#1}}
    % Ensure that by default, figures have no caption (until we provide a
    % proper Figure object with a Caption API and a way to capture that
    % in the conversion process - todo).
    \usepackage{caption}
    \DeclareCaptionLabelFormat{nolabel}{}
    \captionsetup{labelformat=nolabel}

    \usepackage{adjustbox} % Used to constrain images to a maximum size
    \usepackage{xcolor} % Allow colors to be defined
    \usepackage{enumerate} % Needed for markdown enumerations to work
    \usepackage{geometry} % Used to adjust the document margins
    \usepackage{amsmath} % Equations
    \usepackage{amssymb} % Equations
    \usepackage{textcomp} % defines textquotesingle
    % Hack from http://tex.stackexchange.com/a/47451/13684:
    \AtBeginDocument{%
        \def\PYZsq{\textquotesingle}% Upright quotes in Pygmentized code
    }
    \usepackage{upquote} % Upright quotes for verbatim code
    \usepackage{eurosym} % defines \euro
    \usepackage[mathletters]{ucs} % Extended unicode (utf-8) support
    \usepackage[utf8x]{inputenc} % Allow utf-8 characters in the tex document
    \usepackage{fancyvrb} % verbatim replacement that allows latex
    \usepackage{grffile} % extends the file name processing of package graphics
                         % to support a larger range
    % The hyperref package gives us a pdf with properly built
    % internal navigation ('pdf bookmarks' for the table of contents,
    % internal cross-reference links, web links for URLs, etc.)
    \usepackage{hyperref}
    \usepackage{longtable} % longtable support required by pandoc >1.10
    \usepackage{booktabs}  % table support for pandoc > 1.12.2
    \usepackage[inline]{enumitem} % IRkernel/repr support (it uses the enumerate* environment)
    \usepackage[normalem]{ulem} % ulem is needed to support strikethroughs (\sout)
                                % normalem makes italics be italics, not underlines




    % Colors for the hyperref package
    \definecolor{urlcolor}{rgb}{0,.145,.698}
    \definecolor{linkcolor}{rgb}{.71,0.21,0.01}
    \definecolor{citecolor}{rgb}{.12,.54,.11}

    % ANSI colors
    \definecolor{ansi-black}{HTML}{3E424D}
    \definecolor{ansi-black-intense}{HTML}{282C36}
    \definecolor{ansi-red}{HTML}{E75C58}
    \definecolor{ansi-red-intense}{HTML}{B22B31}
    \definecolor{ansi-green}{HTML}{00A250}
    \definecolor{ansi-green-intense}{HTML}{007427}
    \definecolor{ansi-yellow}{HTML}{DDB62B}
    \definecolor{ansi-yellow-intense}{HTML}{B27D12}
    \definecolor{ansi-blue}{HTML}{208FFB}
    \definecolor{ansi-blue-intense}{HTML}{0065CA}
    \definecolor{ansi-magenta}{HTML}{D160C4}
    \definecolor{ansi-magenta-intense}{HTML}{A03196}
    \definecolor{ansi-cyan}{HTML}{60C6C8}
    \definecolor{ansi-cyan-intense}{HTML}{258F8F}
    \definecolor{ansi-white}{HTML}{C5C1B4}
    \definecolor{ansi-white-intense}{HTML}{A1A6B2}

    % commands and environments needed by pandoc snippets
    % extracted from the output of `pandoc -s`
    \providecommand{\tightlist}{%
      \setlength{\itemsep}{0pt}\setlength{\parskip}{0pt}}
    \DefineVerbatimEnvironment{Highlighting}{Verbatim}{commandchars=\\\{\}}
    % Add ',fontsize=\small' for more characters per line
    \newenvironment{Shaded}{}{}
    \newcommand{\KeywordTok}[1]{\textcolor[rgb]{0.00,0.44,0.13}{\textbf{{#1}}}}
    \newcommand{\DataTypeTok}[1]{\textcolor[rgb]{0.56,0.13,0.00}{{#1}}}
    \newcommand{\DecValTok}[1]{\textcolor[rgb]{0.25,0.63,0.44}{{#1}}}
    \newcommand{\BaseNTok}[1]{\textcolor[rgb]{0.25,0.63,0.44}{{#1}}}
    \newcommand{\FloatTok}[1]{\textcolor[rgb]{0.25,0.63,0.44}{{#1}}}
    \newcommand{\CharTok}[1]{\textcolor[rgb]{0.25,0.44,0.63}{{#1}}}
    \newcommand{\StringTok}[1]{\textcolor[rgb]{0.25,0.44,0.63}{{#1}}}
    \newcommand{\CommentTok}[1]{\textcolor[rgb]{0.38,0.63,0.69}{\textit{{#1}}}}
    \newcommand{\OtherTok}[1]{\textcolor[rgb]{0.00,0.44,0.13}{{#1}}}
    \newcommand{\AlertTok}[1]{\textcolor[rgb]{1.00,0.00,0.00}{\textbf{{#1}}}}
    \newcommand{\FunctionTok}[1]{\textcolor[rgb]{0.02,0.16,0.49}{{#1}}}
    \newcommand{\RegionMarkerTok}[1]{{#1}}
    \newcommand{\ErrorTok}[1]{\textcolor[rgb]{1.00,0.00,0.00}{\textbf{{#1}}}}
    \newcommand{\NormalTok}[1]{{#1}}

    % Additional commands for more recent versions of Pandoc
    \newcommand{\ConstantTok}[1]{\textcolor[rgb]{0.53,0.00,0.00}{{#1}}}
    \newcommand{\SpecialCharTok}[1]{\textcolor[rgb]{0.25,0.44,0.63}{{#1}}}
    \newcommand{\VerbatimStringTok}[1]{\textcolor[rgb]{0.25,0.44,0.63}{{#1}}}
    \newcommand{\SpecialStringTok}[1]{\textcolor[rgb]{0.73,0.40,0.53}{{#1}}}
    \newcommand{\ImportTok}[1]{{#1}}
    \newcommand{\DocumentationTok}[1]{\textcolor[rgb]{0.73,0.13,0.13}{\textit{{#1}}}}
    \newcommand{\AnnotationTok}[1]{\textcolor[rgb]{0.38,0.63,0.69}{\textbf{\textit{{#1}}}}}
    \newcommand{\CommentVarTok}[1]{\textcolor[rgb]{0.38,0.63,0.69}{\textbf{\textit{{#1}}}}}
    \newcommand{\VariableTok}[1]{\textcolor[rgb]{0.10,0.09,0.49}{{#1}}}
    \newcommand{\ControlFlowTok}[1]{\textcolor[rgb]{0.00,0.44,0.13}{\textbf{{#1}}}}
    \newcommand{\OperatorTok}[1]{\textcolor[rgb]{0.40,0.40,0.40}{{#1}}}
    \newcommand{\BuiltInTok}[1]{{#1}}
    \newcommand{\ExtensionTok}[1]{{#1}}
    \newcommand{\PreprocessorTok}[1]{\textcolor[rgb]{0.74,0.48,0.00}{{#1}}}
    \newcommand{\AttributeTok}[1]{\textcolor[rgb]{0.49,0.56,0.16}{{#1}}}
    \newcommand{\InformationTok}[1]{\textcolor[rgb]{0.38,0.63,0.69}{\textbf{\textit{{#1}}}}}
    \newcommand{\WarningTok}[1]{\textcolor[rgb]{0.38,0.63,0.69}{\textbf{\textit{{#1}}}}}


    % Define a nice break command that doesn't care if a line doesn't already
    % exist.
    \def\br{\hspace*{\fill} \\* }
    % Math Jax compatability definitions
    \def\gt{>}
    \def\lt{<}
    % Document parameters
    \title{MobileData\_solutions}




    % Pygments definitions

\makeatletter
\def\PY@reset{\let\PY@it=\relax \let\PY@bf=\relax%
    \let\PY@ul=\relax \let\PY@tc=\relax%
    \let\PY@bc=\relax \let\PY@ff=\relax}
\def\PY@tok#1{\csname PY@tok@#1\endcsname}
\def\PY@toks#1+{\ifx\relax#1\empty\else%
    \PY@tok{#1}\expandafter\PY@toks\fi}
\def\PY@do#1{\PY@bc{\PY@tc{\PY@ul{%
    \PY@it{\PY@bf{\PY@ff{#1}}}}}}}
\def\PY#1#2{\PY@reset\PY@toks#1+\relax+\PY@do{#2}}

\expandafter\def\csname PY@tok@w\endcsname{\def\PY@tc##1{\textcolor[rgb]{0.73,0.73,0.73}{##1}}}
\expandafter\def\csname PY@tok@c\endcsname{\let\PY@it=\textit\def\PY@tc##1{\textcolor[rgb]{0.25,0.50,0.50}{##1}}}
\expandafter\def\csname PY@tok@cp\endcsname{\def\PY@tc##1{\textcolor[rgb]{0.74,0.48,0.00}{##1}}}
\expandafter\def\csname PY@tok@k\endcsname{\let\PY@bf=\textbf\def\PY@tc##1{\textcolor[rgb]{0.00,0.50,0.00}{##1}}}
\expandafter\def\csname PY@tok@kp\endcsname{\def\PY@tc##1{\textcolor[rgb]{0.00,0.50,0.00}{##1}}}
\expandafter\def\csname PY@tok@kt\endcsname{\def\PY@tc##1{\textcolor[rgb]{0.69,0.00,0.25}{##1}}}
\expandafter\def\csname PY@tok@o\endcsname{\def\PY@tc##1{\textcolor[rgb]{0.40,0.40,0.40}{##1}}}
\expandafter\def\csname PY@tok@ow\endcsname{\let\PY@bf=\textbf\def\PY@tc##1{\textcolor[rgb]{0.67,0.13,1.00}{##1}}}
\expandafter\def\csname PY@tok@nb\endcsname{\def\PY@tc##1{\textcolor[rgb]{0.00,0.50,0.00}{##1}}}
\expandafter\def\csname PY@tok@nf\endcsname{\def\PY@tc##1{\textcolor[rgb]{0.00,0.00,1.00}{##1}}}
\expandafter\def\csname PY@tok@nc\endcsname{\let\PY@bf=\textbf\def\PY@tc##1{\textcolor[rgb]{0.00,0.00,1.00}{##1}}}
\expandafter\def\csname PY@tok@nn\endcsname{\let\PY@bf=\textbf\def\PY@tc##1{\textcolor[rgb]{0.00,0.00,1.00}{##1}}}
\expandafter\def\csname PY@tok@ne\endcsname{\let\PY@bf=\textbf\def\PY@tc##1{\textcolor[rgb]{0.82,0.25,0.23}{##1}}}
\expandafter\def\csname PY@tok@nv\endcsname{\def\PY@tc##1{\textcolor[rgb]{0.10,0.09,0.49}{##1}}}
\expandafter\def\csname PY@tok@no\endcsname{\def\PY@tc##1{\textcolor[rgb]{0.53,0.00,0.00}{##1}}}
\expandafter\def\csname PY@tok@nl\endcsname{\def\PY@tc##1{\textcolor[rgb]{0.63,0.63,0.00}{##1}}}
\expandafter\def\csname PY@tok@ni\endcsname{\let\PY@bf=\textbf\def\PY@tc##1{\textcolor[rgb]{0.60,0.60,0.60}{##1}}}
\expandafter\def\csname PY@tok@na\endcsname{\def\PY@tc##1{\textcolor[rgb]{0.49,0.56,0.16}{##1}}}
\expandafter\def\csname PY@tok@nt\endcsname{\let\PY@bf=\textbf\def\PY@tc##1{\textcolor[rgb]{0.00,0.50,0.00}{##1}}}
\expandafter\def\csname PY@tok@nd\endcsname{\def\PY@tc##1{\textcolor[rgb]{0.67,0.13,1.00}{##1}}}
\expandafter\def\csname PY@tok@s\endcsname{\def\PY@tc##1{\textcolor[rgb]{0.73,0.13,0.13}{##1}}}
\expandafter\def\csname PY@tok@sd\endcsname{\let\PY@it=\textit\def\PY@tc##1{\textcolor[rgb]{0.73,0.13,0.13}{##1}}}
\expandafter\def\csname PY@tok@si\endcsname{\let\PY@bf=\textbf\def\PY@tc##1{\textcolor[rgb]{0.73,0.40,0.53}{##1}}}
\expandafter\def\csname PY@tok@se\endcsname{\let\PY@bf=\textbf\def\PY@tc##1{\textcolor[rgb]{0.73,0.40,0.13}{##1}}}
\expandafter\def\csname PY@tok@sr\endcsname{\def\PY@tc##1{\textcolor[rgb]{0.73,0.40,0.53}{##1}}}
\expandafter\def\csname PY@tok@ss\endcsname{\def\PY@tc##1{\textcolor[rgb]{0.10,0.09,0.49}{##1}}}
\expandafter\def\csname PY@tok@sx\endcsname{\def\PY@tc##1{\textcolor[rgb]{0.00,0.50,0.00}{##1}}}
\expandafter\def\csname PY@tok@m\endcsname{\def\PY@tc##1{\textcolor[rgb]{0.40,0.40,0.40}{##1}}}
\expandafter\def\csname PY@tok@gh\endcsname{\let\PY@bf=\textbf\def\PY@tc##1{\textcolor[rgb]{0.00,0.00,0.50}{##1}}}
\expandafter\def\csname PY@tok@gu\endcsname{\let\PY@bf=\textbf\def\PY@tc##1{\textcolor[rgb]{0.50,0.00,0.50}{##1}}}
\expandafter\def\csname PY@tok@gd\endcsname{\def\PY@tc##1{\textcolor[rgb]{0.63,0.00,0.00}{##1}}}
\expandafter\def\csname PY@tok@gi\endcsname{\def\PY@tc##1{\textcolor[rgb]{0.00,0.63,0.00}{##1}}}
\expandafter\def\csname PY@tok@gr\endcsname{\def\PY@tc##1{\textcolor[rgb]{1.00,0.00,0.00}{##1}}}
\expandafter\def\csname PY@tok@ge\endcsname{\let\PY@it=\textit}
\expandafter\def\csname PY@tok@gs\endcsname{\let\PY@bf=\textbf}
\expandafter\def\csname PY@tok@gp\endcsname{\let\PY@bf=\textbf\def\PY@tc##1{\textcolor[rgb]{0.00,0.00,0.50}{##1}}}
\expandafter\def\csname PY@tok@go\endcsname{\def\PY@tc##1{\textcolor[rgb]{0.53,0.53,0.53}{##1}}}
\expandafter\def\csname PY@tok@gt\endcsname{\def\PY@tc##1{\textcolor[rgb]{0.00,0.27,0.87}{##1}}}
\expandafter\def\csname PY@tok@err\endcsname{\def\PY@bc##1{\setlength{\fboxsep}{0pt}\fcolorbox[rgb]{1.00,0.00,0.00}{1,1,1}{\strut ##1}}}
\expandafter\def\csname PY@tok@kc\endcsname{\let\PY@bf=\textbf\def\PY@tc##1{\textcolor[rgb]{0.00,0.50,0.00}{##1}}}
\expandafter\def\csname PY@tok@kd\endcsname{\let\PY@bf=\textbf\def\PY@tc##1{\textcolor[rgb]{0.00,0.50,0.00}{##1}}}
\expandafter\def\csname PY@tok@kn\endcsname{\let\PY@bf=\textbf\def\PY@tc##1{\textcolor[rgb]{0.00,0.50,0.00}{##1}}}
\expandafter\def\csname PY@tok@kr\endcsname{\let\PY@bf=\textbf\def\PY@tc##1{\textcolor[rgb]{0.00,0.50,0.00}{##1}}}
\expandafter\def\csname PY@tok@bp\endcsname{\def\PY@tc##1{\textcolor[rgb]{0.00,0.50,0.00}{##1}}}
\expandafter\def\csname PY@tok@fm\endcsname{\def\PY@tc##1{\textcolor[rgb]{0.00,0.00,1.00}{##1}}}
\expandafter\def\csname PY@tok@vc\endcsname{\def\PY@tc##1{\textcolor[rgb]{0.10,0.09,0.49}{##1}}}
\expandafter\def\csname PY@tok@vg\endcsname{\def\PY@tc##1{\textcolor[rgb]{0.10,0.09,0.49}{##1}}}
\expandafter\def\csname PY@tok@vi\endcsname{\def\PY@tc##1{\textcolor[rgb]{0.10,0.09,0.49}{##1}}}
\expandafter\def\csname PY@tok@vm\endcsname{\def\PY@tc##1{\textcolor[rgb]{0.10,0.09,0.49}{##1}}}
\expandafter\def\csname PY@tok@sa\endcsname{\def\PY@tc##1{\textcolor[rgb]{0.73,0.13,0.13}{##1}}}
\expandafter\def\csname PY@tok@sb\endcsname{\def\PY@tc##1{\textcolor[rgb]{0.73,0.13,0.13}{##1}}}
\expandafter\def\csname PY@tok@sc\endcsname{\def\PY@tc##1{\textcolor[rgb]{0.73,0.13,0.13}{##1}}}
\expandafter\def\csname PY@tok@dl\endcsname{\def\PY@tc##1{\textcolor[rgb]{0.73,0.13,0.13}{##1}}}
\expandafter\def\csname PY@tok@s2\endcsname{\def\PY@tc##1{\textcolor[rgb]{0.73,0.13,0.13}{##1}}}
\expandafter\def\csname PY@tok@sh\endcsname{\def\PY@tc##1{\textcolor[rgb]{0.73,0.13,0.13}{##1}}}
\expandafter\def\csname PY@tok@s1\endcsname{\def\PY@tc##1{\textcolor[rgb]{0.73,0.13,0.13}{##1}}}
\expandafter\def\csname PY@tok@mb\endcsname{\def\PY@tc##1{\textcolor[rgb]{0.40,0.40,0.40}{##1}}}
\expandafter\def\csname PY@tok@mf\endcsname{\def\PY@tc##1{\textcolor[rgb]{0.40,0.40,0.40}{##1}}}
\expandafter\def\csname PY@tok@mh\endcsname{\def\PY@tc##1{\textcolor[rgb]{0.40,0.40,0.40}{##1}}}
\expandafter\def\csname PY@tok@mi\endcsname{\def\PY@tc##1{\textcolor[rgb]{0.40,0.40,0.40}{##1}}}
\expandafter\def\csname PY@tok@il\endcsname{\def\PY@tc##1{\textcolor[rgb]{0.40,0.40,0.40}{##1}}}
\expandafter\def\csname PY@tok@mo\endcsname{\def\PY@tc##1{\textcolor[rgb]{0.40,0.40,0.40}{##1}}}
\expandafter\def\csname PY@tok@ch\endcsname{\let\PY@it=\textit\def\PY@tc##1{\textcolor[rgb]{0.25,0.50,0.50}{##1}}}
\expandafter\def\csname PY@tok@cm\endcsname{\let\PY@it=\textit\def\PY@tc##1{\textcolor[rgb]{0.25,0.50,0.50}{##1}}}
\expandafter\def\csname PY@tok@cpf\endcsname{\let\PY@it=\textit\def\PY@tc##1{\textcolor[rgb]{0.25,0.50,0.50}{##1}}}
\expandafter\def\csname PY@tok@c1\endcsname{\let\PY@it=\textit\def\PY@tc##1{\textcolor[rgb]{0.25,0.50,0.50}{##1}}}
\expandafter\def\csname PY@tok@cs\endcsname{\let\PY@it=\textit\def\PY@tc##1{\textcolor[rgb]{0.25,0.50,0.50}{##1}}}

\def\PYZbs{\char`\\}
\def\PYZus{\char`\_}
\def\PYZob{\char`\{}
\def\PYZcb{\char`\}}
\def\PYZca{\char`\^}
\def\PYZam{\char`\&}
\def\PYZlt{\char`\<}
\def\PYZgt{\char`\>}
\def\PYZsh{\char`\#}
\def\PYZpc{\char`\%}
\def\PYZdl{\char`\$}
\def\PYZhy{\char`\-}
\def\PYZsq{\char`\'}
\def\PYZdq{\char`\"}
\def\PYZti{\char`\~}
% for compatibility with earlier versions
\def\PYZat{@}
\def\PYZlb{[}
\def\PYZrb{]}
\makeatother


    % Exact colors from NB
    \definecolor{incolor}{rgb}{0.0, 0.0, 0.5}
    \definecolor{outcolor}{rgb}{0.545, 0.0, 0.0}




    % Prevent overflowing lines due to hard-to-break entities
    \sloppy
    % Setup hyperref package
    \hypersetup{
      breaklinks=true,  % so long urls are correctly broken across lines
      colorlinks=true,
      urlcolor=urlcolor,
      linkcolor=linkcolor,
      citecolor=citecolor,
      }
    % Slightly bigger margins than the latex defaults

    \geometry{verbose,tmargin=1in,bmargin=1in,lmargin=1in,rmargin=1in}



    \begin{document}


    \maketitle




    In this notebook we will build our own supervised machine learning
algorithm from the bottom and up. The approach taken here is based on
the general recipe introduced in the slides. We make the following
choices - Model: logistic regression - Cost function: the cross entropy
between the predictions and the empirical data - Optimization algorithm:
stochastic gradient descent

We will apply this framework to predict if a person is going to be on
the move fifteen minutes from now based on the activity of that person
for the last two hours. The information on activity includes calls,
texts, movement, and physical proximity to other people. In this
notebook we will use the same data set for training and testing, but we
note that the same results are obtained with proper cross validation.

    To do our analysis we are going to need numpy for vector manipulations.
Also, we are going to need pickle to load our data.

    \begin{Verbatim}[commandchars=\\\{\}]
{\color{incolor}In [{\color{incolor}1}]:} \PY{k+kn}{import} \PY{n+nn}{numpy} \PY{k}{as} \PY{n+nn}{np}
        \PY{k+kn}{import} \PY{n+nn}{pickle}
\end{Verbatim}


    We then load the data. Start by inserting the path to the directory of
the data

    \begin{Verbatim}[commandchars=\\\{\}]
{\color{incolor}In [{\color{incolor}3}]:} \PY{n}{data\PYZus{}dir} \PY{o}{=} \PY{l+s+s1}{\PYZsq{}}\PY{l+s+s1}{../allan\PYZus{}data/}\PY{l+s+s1}{\PYZsq{}} \PY{c+c1}{\PYZsh{} INSERT YOUR OWN PATH HERE}

        \PY{k}{with} \PY{n+nb}{open}\PY{p}{(}\PY{l+s+s1}{\PYZsq{}}\PY{l+s+si}{\PYZpc{}s}\PY{l+s+s1}{/DataPredictMovement\PYZus{}half.p}\PY{l+s+s1}{\PYZsq{}} \PY{o}{\PYZpc{}}\PY{k}{data\PYZus{}dir}, \PYZsq{}rb\PYZsq{}) as f:
            \PY{n}{X}\PY{p}{,} \PY{n}{Y} \PY{o}{=} \PY{n}{pickle}\PY{o}{.}\PY{n}{load}\PY{p}{(}\PY{n}{f}\PY{p}{)}
\end{Verbatim}


    X and Y are numpy arrays with each row corresponding to a data point.
The rows of Y contain just a single number, which tells us whether there
is movement 15 minutes into the future or not. A positive answer is a 1
and a negative answer is a -1. Each row of X contains 32 numbers, which
tells us if there has been activity for the past 8 quarters in any of
the 4 activity channels (calls, texts, movement, social proximity). This
is likewise answered with 1's and -1's.

    \begin{Verbatim}[commandchars=\\\{\}]
{\color{incolor}In [{\color{incolor}4}]:} \PY{n+nb}{print}\PY{p}{(}\PY{l+s+s1}{\PYZsq{}}\PY{l+s+s1}{The shape of X:}\PY{l+s+s1}{\PYZsq{}}\PY{p}{,} \PY{n}{X}\PY{o}{.}\PY{n}{shape}\PY{p}{)}
        \PY{n+nb}{print}\PY{p}{(}\PY{l+s+s1}{\PYZsq{}}\PY{l+s+s1}{The shape of Y:}\PY{l+s+s1}{\PYZsq{}}\PY{p}{,} \PY{n}{Y}\PY{o}{.}\PY{n}{shape}\PY{p}{)}
        \PY{n+nb}{print}\PY{p}{(}\PY{l+s+s1}{\PYZsq{}}\PY{l+s+s1}{The first row of X:}\PY{l+s+s1}{\PYZsq{}}\PY{p}{,} \PY{n}{X}\PY{p}{[}\PY{l+m+mi}{0}\PY{p}{]}\PY{p}{)}
        \PY{n+nb}{print}\PY{p}{(}\PY{l+s+s1}{\PYZsq{}}\PY{l+s+s1}{The first row of Y:}\PY{l+s+s1}{\PYZsq{}}\PY{p}{,} \PY{n}{Y}\PY{p}{[}\PY{l+m+mi}{0}\PY{p}{]}\PY{p}{)}
\end{Verbatim}


    \begin{Verbatim}[commandchars=\\\{\}]
The shape of X: (6549096, 32)
The shape of Y: (6549096,)
The first row of X: [-1 -1 -1  1 -1 -1  1 -1 -1 -1  1 -1 -1 -1 -1 -1 -1 -1 -1 -1 -1 -1 -1 -1
 -1 -1 -1 -1 -1 -1 -1 -1]
The first row of Y: -1

    \end{Verbatim}

    Now that we have loaded and familiarized ourselves with the data it is
time to do some machine learning. Our objective is to predict Y given X.
As noted above, we are going to build our own algorithm based on
logistic regression, cross entropy, and stochastic gradient descent.
These are defined in the slides and we now define them in our program.

    Exercise: I have defined the function for the gradient at the bottom of
the following code. You must now fill out the functions for the support,
the model, and the cost. You should furthermore fill out the update
function such that it returns updated parameters. You can find the
mathematics in the slides and you can guide yourselves from the input
and output already defined in the code. You will need some or all of the
following numpy functions: np.sum(), np.dot(), np.exp(), and np.log().

    \begin{Verbatim}[commandchars=\\\{\}]
{\color{incolor}In [{\color{incolor}5}]:} \PY{k}{def} \PY{n+nf}{calc\PYZus{}support}\PY{p}{(}\PY{n}{X}\PY{p}{,} \PY{n}{W}\PY{p}{,} \PY{n}{b}\PY{p}{)}\PY{p}{:}
            \PY{l+s+sd}{\PYZsq{}\PYZsq{}\PYZsq{}Returns the support for movement (Y=1) based on the input (X) }
        \PY{l+s+sd}{    and the parameters (W and b)\PYZsq{}\PYZsq{}\PYZsq{}}
            \PY{n}{support} \PY{o}{=} \PY{n}{np}\PY{o}{.}\PY{n}{dot}\PY{p}{(}\PY{n}{X}\PY{p}{,} \PY{n}{W}\PY{p}{)} \PY{o}{+} \PY{n}{b}
            \PY{k}{return} \PY{n}{support}


        \PY{k}{def} \PY{n+nf}{model}\PY{p}{(}\PY{n}{support}\PY{p}{)}\PY{p}{:}
            \PY{l+s+sd}{\PYZsq{}\PYZsq{}\PYZsq{}Returns the predicted probability of movement based on the support.\PYZsq{}\PYZsq{}\PYZsq{}}
            \PY{n}{Y\PYZus{}pred} \PY{o}{=} \PY{l+m+mi}{1} \PY{o}{/} \PY{p}{(}\PY{l+m+mi}{1} \PY{o}{+} \PY{n}{np}\PY{o}{.}\PY{n}{exp}\PY{p}{(}\PY{o}{\PYZhy{}}\PY{n}{support}\PY{p}{)}\PY{p}{)}
            \PY{k}{return} \PY{n}{Y\PYZus{}pred}


        \PY{k}{def} \PY{n+nf}{calc\PYZus{}cost}\PY{p}{(}\PY{n}{support}\PY{p}{,} \PY{n}{Y}\PY{p}{)}\PY{p}{:}
            \PY{l+s+sd}{\PYZsq{}\PYZsq{}\PYZsq{}Returns the cost of our predictions, i.e. how much our predictions failed.\PYZsq{}\PYZsq{}\PYZsq{}}
            \PY{n}{N} \PY{o}{=} \PY{n}{Y}\PY{o}{.}\PY{n}{shape}\PY{p}{[}\PY{l+m+mi}{0}\PY{p}{]}
            \PY{n}{cost} \PY{o}{=} \PY{n}{np}\PY{o}{.}\PY{n}{log}\PY{p}{(}\PY{l+m+mi}{1} \PY{o}{+} \PY{n}{np}\PY{o}{.}\PY{n}{exp}\PY{p}{(}\PY{o}{\PYZhy{}}\PY{n}{support} \PY{o}{*} \PY{n}{Y}\PY{p}{)}\PY{p}{)}\PY{o}{.}\PY{n}{sum}\PY{p}{(}\PY{p}{)} \PY{o}{/} \PY{n}{N}
            \PY{k}{return} \PY{n}{cost}


        \PY{k}{def} \PY{n+nf}{update}\PY{p}{(}\PY{n}{W}\PY{p}{,} \PY{n}{b}\PY{p}{,} \PY{n}{gradient\PYZus{}W}\PY{p}{,} \PY{n}{gradient\PYZus{}b}\PY{p}{,} \PY{n}{learning\PYZus{}rate}\PY{o}{=}\PY{l+m+mf}{0.1}\PY{p}{)}\PY{p}{:}
            \PY{l+s+sd}{\PYZsq{}\PYZsq{}\PYZsq{}Returns updated parameters based on the gradients. The learning rate}
        \PY{l+s+sd}{    determines how far we move along the direction of the gradient.\PYZsq{}\PYZsq{}\PYZsq{}}
            \PY{n}{W} \PY{o}{=} \PY{n}{W} \PY{o}{\PYZhy{}} \PY{n}{learning\PYZus{}rate} \PY{o}{*} \PY{n}{gradient\PYZus{}W}
            \PY{n}{b} \PY{o}{=} \PY{n}{b} \PY{o}{\PYZhy{}} \PY{n}{learning\PYZus{}rate} \PY{o}{*} \PY{n}{gradient\PYZus{}b}
            \PY{k}{return} \PY{n}{W}\PY{p}{,} \PY{n}{b}


        \PY{k}{def} \PY{n+nf}{calc\PYZus{}gradient}\PY{p}{(}\PY{n}{support}\PY{p}{,} \PY{n}{X}\PY{p}{,} \PY{n}{Y}\PY{p}{)}\PY{p}{:}
            \PY{l+s+sd}{\PYZsq{}\PYZsq{}\PYZsq{}Returns the gradient of the cost with respect to the parameters. This}
        \PY{l+s+sd}{    determines the direction in parameter space that will bring down the cost.\PYZsq{}\PYZsq{}\PYZsq{}}
            \PY{n}{N} \PY{o}{=} \PY{n}{Y}\PY{o}{.}\PY{n}{shape}\PY{p}{[}\PY{l+m+mi}{0}\PY{p}{]}
            \PY{n}{gradient\PYZus{}W} \PY{o}{=} \PY{n}{np}\PY{o}{.}\PY{n}{dot}\PY{p}{(} \PY{o}{\PYZhy{}}\PY{n}{Y} \PY{o}{/} \PY{p}{(}\PY{l+m+mi}{1} \PY{o}{+} \PY{n}{np}\PY{o}{.}\PY{n}{exp}\PY{p}{(} \PY{n}{Y} \PY{o}{*} \PY{n}{support} \PY{p}{)}\PY{p}{)} \PY{o}{/} \PY{n}{N}\PY{p}{,} \PY{n}{X}\PY{p}{)}
            \PY{n}{gradient\PYZus{}b} \PY{o}{=} \PY{n}{np}\PY{o}{.}\PY{n}{sum}\PY{p}{(} \PY{o}{\PYZhy{}}\PY{n}{Y} \PY{o}{/} \PY{p}{(}\PY{l+m+mi}{1} \PY{o}{+} \PY{n}{np}\PY{o}{.}\PY{n}{exp}\PY{p}{(} \PY{n}{Y} \PY{o}{*} \PY{n}{support} \PY{p}{)}\PY{p}{)} \PY{o}{/} \PY{n}{N} \PY{p}{)}
            \PY{k}{return} \PY{n}{gradient\PYZus{}W}\PY{p}{,} \PY{n}{gradient\PYZus{}b}
\end{Verbatim}


    Using the above functions we can now make predictions (using model), we
can measure how good these predictions are (using calc\_cost), and we
can update our parameters to make the predictions better (using update).
So now we just need to initialize our parameters (W and b) and then
follow the gradient during a lot of updates to minimize the cost.

    \begin{Verbatim}[commandchars=\\\{\}]
{\color{incolor}In [{\color{incolor}6}]:} \PY{c+c1}{\PYZsh{} initialize the variables}
        \PY{n}{W} \PY{o}{=} \PY{n}{np}\PY{o}{.}\PY{n}{random}\PY{o}{.}\PY{n}{normal}\PY{p}{(}\PY{l+m+mi}{0}\PY{p}{,}\PY{l+m+mi}{1}\PY{p}{,}\PY{n}{X}\PY{o}{.}\PY{n}{shape}\PY{p}{[}\PY{l+m+mi}{1}\PY{p}{]}\PY{p}{)}
        \PY{n}{b} \PY{o}{=} \PY{l+m+mf}{0.}

        \PY{c+c1}{\PYZsh{} set the parameters of the optimization algorithm}
        \PY{n}{learning\PYZus{}rate} \PY{o}{=} \PY{l+m+mf}{0.01}
        \PY{n}{batch\PYZus{}size} \PY{o}{=} \PY{l+m+mi}{100}
        \PY{n}{steps} \PY{o}{=} \PY{l+m+mi}{100000}

        \PY{c+c1}{\PYZsh{} follow the gradient!}
        \PY{k}{for} \PY{n}{step} \PY{o+ow}{in} \PY{n+nb}{range}\PY{p}{(}\PY{n}{steps}\PY{p}{)}\PY{p}{:}

            \PY{c+c1}{\PYZsh{} we randomly samples batches of data points}
            \PY{n}{batch\PYZus{}indices} \PY{o}{=} \PY{n}{np}\PY{o}{.}\PY{n}{random}\PY{o}{.}\PY{n}{choice}\PY{p}{(}\PY{n}{Y}\PY{o}{.}\PY{n}{shape}\PY{p}{[}\PY{l+m+mi}{0}\PY{p}{]}\PY{p}{,} \PY{n}{batch\PYZus{}size}\PY{p}{)}
            \PY{n}{X\PYZus{}batch}\PY{p}{,} \PY{n}{Y\PYZus{}batch} \PY{o}{=} \PY{n}{X}\PY{p}{[}\PY{n}{batch\PYZus{}indices}\PY{p}{]}\PY{p}{,} \PY{n}{Y}\PY{p}{[}\PY{n}{batch\PYZus{}indices}\PY{p}{]}

            \PY{c+c1}{\PYZsh{} we calculate the cost on the full dataset every 10000 step to check the progress}
            \PY{c+c1}{\PYZsh{} note: this is the computationally expensive part. It can be left out since it}
            \PY{c+c1}{\PYZsh{} does not affect the training.}
            \PY{k}{if} \PY{n}{step} \PY{o}{\PYZpc{}} \PY{p}{(}\PY{n}{steps}\PY{o}{/}\PY{o}{/}\PY{l+m+mi}{10}\PY{p}{)} \PY{o}{==} \PY{l+m+mi}{0}\PY{p}{:}
                \PY{n}{support} \PY{o}{=} \PY{n}{calc\PYZus{}support}\PY{p}{(}\PY{n}{X}\PY{p}{,} \PY{n}{W}\PY{p}{,} \PY{n}{b}\PY{p}{)}
                \PY{n}{cost} \PY{o}{=} \PY{n}{calc\PYZus{}cost}\PY{p}{(}\PY{n}{support}\PY{p}{,} \PY{n}{Y}\PY{p}{)}
                \PY{n+nb}{print}\PY{p}{(}\PY{l+s+s1}{\PYZsq{}}\PY{l+s+s1}{\PYZsh{} Cost at step }\PY{l+s+si}{\PYZpc{}d}\PY{l+s+s1}{: }\PY{l+s+si}{\PYZpc{}f}\PY{l+s+s1}{\PYZsq{}} \PY{o}{\PYZpc{}}\PY{p}{(}\PY{n}{step}\PY{p}{,} \PY{n}{cost}\PY{p}{)}\PY{p}{)}

            \PY{c+c1}{\PYZsh{} we update the parameters at each step}
            \PY{n}{support\PYZus{}batch} \PY{o}{=} \PY{n}{calc\PYZus{}support}\PY{p}{(}\PY{n}{X\PYZus{}batch}\PY{p}{,} \PY{n}{W}\PY{p}{,} \PY{n}{b}\PY{p}{)}
            \PY{n}{gradient\PYZus{}W}\PY{p}{,} \PY{n}{gradient\PYZus{}b} \PY{o}{=} \PY{n}{calc\PYZus{}gradient}\PY{p}{(}\PY{n}{support\PYZus{}batch}\PY{p}{,} \PY{n}{X\PYZus{}batch}\PY{p}{,} \PY{n}{Y\PYZus{}batch}\PY{p}{)}
            \PY{n}{W}\PY{p}{,} \PY{n}{b} \PY{o}{=} \PY{n}{update}\PY{p}{(}\PY{n}{W}\PY{p}{,} \PY{n}{b}\PY{p}{,} \PY{n}{gradient\PYZus{}W}\PY{p}{,} \PY{n}{gradient\PYZus{}b}\PY{p}{,} \PY{n}{learning\PYZus{}rate}\PY{p}{)}
\end{Verbatim}


    \begin{Verbatim}[commandchars=\\\{\}]
\# Cost at step 0: 1.089153
\# Cost at step 10000: 0.235301
\# Cost at step 20000: 0.220090
\# Cost at step 30000: 0.216831
\# Cost at step 40000: 0.215953
\# Cost at step 50000: 0.215676
\# Cost at step 60000: 0.215645
\# Cost at step 70000: 0.215522
\# Cost at step 80000: 0.215523
\# Cost at step 90000: 0.215668

    \end{Verbatim}

    We see that the cost drops drastically during the first steps of the
algorithm and then slowly converges. The parameters W and b now have
values that approximately minimize the cost of our predictions.

    Exercise: play around with the parameters of the optimization algorithm
and observe the effects on cost and training time.

    To get a more interpretable measure of the quality of our predictions,
we can calculate the correlation between our predictions probabilities
Y\_pred (numbers from 0 to 1) and the actual values Y (-1's or +1's)

    \begin{Verbatim}[commandchars=\\\{\}]
{\color{incolor}In [{\color{incolor}7}]:} \PY{n}{support} \PY{o}{=} \PY{n}{calc\PYZus{}support}\PY{p}{(}\PY{n}{X}\PY{p}{,} \PY{n}{W}\PY{p}{,} \PY{n}{b}\PY{p}{)}
        \PY{n}{Y\PYZus{}pred} \PY{o}{=} \PY{n}{model}\PY{p}{(}\PY{n}{support}\PY{p}{)}
        \PY{n+nb}{print}\PY{p}{(}\PY{l+s+s1}{\PYZsq{}}\PY{l+s+s1}{Correlation:}\PY{l+s+s1}{\PYZsq{}}\PY{p}{,} \PY{n}{np}\PY{o}{.}\PY{n}{corrcoef}\PY{p}{(}\PY{n}{Y\PYZus{}pred}\PY{p}{,} \PY{n}{Y}\PY{p}{)}\PY{p}{[}\PY{l+m+mi}{0}\PY{p}{,}\PY{l+m+mi}{1}\PY{p}{]}\PY{p}{)}
\end{Verbatim}


    \begin{Verbatim}[commandchars=\\\{\}]
Correlation: 0.5317297225585312

    \end{Verbatim}

    Generally, our predictions yield relatively low probabilities for
movement. This is because people are not moving most of the time in the
actual data

    \begin{Verbatim}[commandchars=\\\{\}]
{\color{incolor}In [{\color{incolor}8}]:} \PY{n+nb}{print}\PY{p}{(}\PY{l+s+s1}{\PYZsq{}}\PY{l+s+s1}{Average probability in predictions:}\PY{l+s+s1}{\PYZsq{}}\PY{p}{,} \PY{n}{np}\PY{o}{.}\PY{n}{mean}\PY{p}{(}\PY{n}{Y\PYZus{}pred}\PY{p}{)}\PY{p}{)}
        \PY{n+nb}{print}\PY{p}{(}\PY{l+s+s1}{\PYZsq{}}\PY{l+s+s1}{Ratio of movement in data:}\PY{l+s+s1}{\PYZsq{}}\PY{p}{,} \PY{p}{(}\PY{n}{Y}\PY{o}{==}\PY{l+m+mi}{1}\PY{p}{)}\PY{o}{.}\PY{n}{sum}\PY{p}{(}\PY{p}{)} \PY{o}{/} \PY{n}{Y}\PY{o}{.}\PY{n}{shape}\PY{p}{[}\PY{l+m+mi}{0}\PY{p}{]}\PY{p}{)}
\end{Verbatim}


    \begin{Verbatim}[commandchars=\\\{\}]
Average probability in predictions: 0.08971108258750721
Ratio of movement in data: 0.09054226720756575

    \end{Verbatim}

    We could get a classification accuracy of 91\% if we simply predicted
``no movement'' for all data points. This illustrates that
classification accuracy is not a very good measure of performance,
especially when dealing with asymmetric data sets. It would be better if
we could measure how well we predict one outcome, while also not failing
on the other outcome. To do this we measure ``true positive rate''
together with the ``false positive rate''. These values depend on the
threshold that we apply to the probabilities to decide whether a data
point should be labeled as positive or negative. In a ROC-curve analysis
we test out several different thresholds to see how each performs. We
will import metric from sklearn to perform this analysis for us and then
use pyplot to plot the results

    \begin{Verbatim}[commandchars=\\\{\}]
{\color{incolor}In [{\color{incolor}9}]:} \PY{k+kn}{from} \PY{n+nn}{sklearn} \PY{k}{import} \PY{n}{metrics}
        \PY{k+kn}{from} \PY{n+nn}{matplotlib} \PY{k}{import} \PY{n}{pyplot} \PY{k}{as} \PY{n}{plt}

        \PY{c+c1}{\PYZsh{} Calculate the roc curve}
        \PY{n}{fpr}\PY{p}{,} \PY{n}{tpr}\PY{p}{,} \PY{n}{thresholds} \PY{o}{=} \PY{n}{metrics}\PY{o}{.}\PY{n}{roc\PYZus{}curve}\PY{p}{(}\PY{n}{Y}\PY{p}{,}\PY{n}{Y\PYZus{}pred}\PY{p}{)}

        \PY{c+c1}{\PYZsh{} Plot the roc curve}
        \PY{n}{plt}\PY{o}{.}\PY{n}{figure}\PY{p}{(}\PY{p}{)}
        \PY{n}{plt}\PY{o}{.}\PY{n}{plot}\PY{p}{(}\PY{n}{fpr}\PY{p}{,} \PY{n}{tpr}\PY{p}{,} \PY{n}{color}\PY{o}{=}\PY{l+s+s1}{\PYZsq{}}\PY{l+s+s1}{darkorange}\PY{l+s+s1}{\PYZsq{}}\PY{p}{,} \PY{n}{lw}\PY{o}{=}\PY{l+m+mi}{2}\PY{p}{,}
                 \PY{n}{label}\PY{o}{=}\PY{l+s+s1}{\PYZsq{}}\PY{l+s+s1}{ROC curve (area = }\PY{l+s+si}{\PYZpc{}0.2f}\PY{l+s+s1}{)}\PY{l+s+s1}{\PYZsq{}} \PY{o}{\PYZpc{}} \PY{n}{metrics}\PY{o}{.}\PY{n}{auc}\PY{p}{(}\PY{n}{fpr}\PY{p}{,} \PY{n}{tpr}\PY{p}{)}\PY{p}{)}
        \PY{n}{plt}\PY{o}{.}\PY{n}{plot}\PY{p}{(}\PY{p}{[}\PY{l+m+mi}{0}\PY{p}{,} \PY{l+m+mi}{1}\PY{p}{]}\PY{p}{,} \PY{p}{[}\PY{l+m+mi}{0}\PY{p}{,} \PY{l+m+mi}{1}\PY{p}{]}\PY{p}{,} \PY{n}{color}\PY{o}{=}\PY{l+s+s1}{\PYZsq{}}\PY{l+s+s1}{navy}\PY{l+s+s1}{\PYZsq{}}\PY{p}{,} \PY{n}{lw}\PY{o}{=}\PY{l+m+mi}{2}\PY{p}{,} \PY{n}{linestyle}\PY{o}{=}\PY{l+s+s1}{\PYZsq{}}\PY{l+s+s1}{\PYZhy{}\PYZhy{}}\PY{l+s+s1}{\PYZsq{}}\PY{p}{)}
        \PY{n}{plt}\PY{o}{.}\PY{n}{xlim}\PY{p}{(}\PY{p}{[}\PY{l+m+mf}{0.0}\PY{p}{,} \PY{l+m+mf}{1.0}\PY{p}{]}\PY{p}{)}
        \PY{n}{plt}\PY{o}{.}\PY{n}{ylim}\PY{p}{(}\PY{p}{[}\PY{l+m+mf}{0.0}\PY{p}{,} \PY{l+m+mf}{1.05}\PY{p}{]}\PY{p}{)}
        \PY{n}{plt}\PY{o}{.}\PY{n}{xlabel}\PY{p}{(}\PY{l+s+s1}{\PYZsq{}}\PY{l+s+s1}{False Positive Rate}\PY{l+s+s1}{\PYZsq{}}\PY{p}{)}
        \PY{n}{plt}\PY{o}{.}\PY{n}{ylabel}\PY{p}{(}\PY{l+s+s1}{\PYZsq{}}\PY{l+s+s1}{True Positive Rate}\PY{l+s+s1}{\PYZsq{}}\PY{p}{)}
        \PY{n}{plt}\PY{o}{.}\PY{n}{legend}\PY{p}{(}\PY{n}{loc}\PY{o}{=}\PY{l+s+s2}{\PYZdq{}}\PY{l+s+s2}{lower right}\PY{l+s+s2}{\PYZdq{}}\PY{p}{)}
        \PY{n}{plt}\PY{o}{.}\PY{n}{show}\PY{p}{(}\PY{p}{)}
\end{Verbatim}



    \begin{verbatim}
<matplotlib.figure.Figure at 0x7eff54e192e8>
    \end{verbatim}


    From the ROC-curve we can see that one of the thresholds allow us to
predict 75\% of the movement, while only getting 20\% false positives.
This is pretty good considering the complexity of human behavior!

    It would also be interesting to check, how the different numbers in the
input vector X add to the support for movement. We do this by splitting
the weight vector into four, one for each type of activity, and then
plot the value of the weights

    \begin{Verbatim}[commandchars=\\\{\}]
{\color{incolor}In [{\color{incolor}10}]:} \PY{c+c1}{\PYZsh{} we separate the weight vector in terms of activity types}
         \PY{n}{activities} \PY{o}{=} \PY{p}{(}\PY{l+s+s1}{\PYZsq{}}\PY{l+s+s1}{call}\PY{l+s+s1}{\PYZsq{}}\PY{p}{,} \PY{l+s+s1}{\PYZsq{}}\PY{l+s+s1}{sms}\PY{l+s+s1}{\PYZsq{}}\PY{p}{,} \PY{l+s+s1}{\PYZsq{}}\PY{l+s+s1}{movement}\PY{l+s+s1}{\PYZsq{}}\PY{p}{,} \PY{l+s+s1}{\PYZsq{}}\PY{l+s+s1}{proximity}\PY{l+s+s1}{\PYZsq{}}\PY{p}{)}
         \PY{n}{W\PYZus{}activities} \PY{o}{=} \PY{p}{\PYZob{}} \PY{n}{activity} \PY{p}{:} \PY{p}{[}\PY{p}{]} \PY{k}{for} \PY{n}{activity} \PY{o+ow}{in} \PY{n}{activities} \PY{p}{\PYZcb{}}
         \PY{k}{for} \PY{n}{ii}\PY{p}{,} \PY{n}{weight} \PY{o+ow}{in} \PY{n+nb}{enumerate}\PY{p}{(}\PY{n}{W}\PY{p}{)}\PY{p}{:}
             \PY{n}{index} \PY{o}{=} \PY{n}{ii} \PY{o}{\PYZpc{}} \PY{l+m+mi}{4}
             \PY{n}{activity} \PY{o}{=} \PY{n}{activities}\PY{p}{[}\PY{n}{index}\PY{p}{]}
             \PY{n}{W\PYZus{}activities}\PY{p}{[}\PY{n}{activity}\PY{p}{]}\PY{o}{.}\PY{n}{insert}\PY{p}{(}\PY{l+m+mi}{0}\PY{p}{,} \PY{n}{weight} \PY{p}{)}

         \PY{c+c1}{\PYZsh{} we plot the weights of each data type}
         \PY{n}{plt}\PY{o}{.}\PY{n}{figure}\PY{p}{(}\PY{p}{)}
         \PY{n}{time} \PY{o}{=} \PY{p}{[} \PY{l+m+mi}{15} \PY{o}{*} \PY{p}{(}\PY{n}{ii} \PY{o}{+} \PY{l+m+mi}{1}\PY{p}{)} \PY{k}{for} \PY{n}{ii} \PY{o+ow}{in} \PY{n+nb}{range}\PY{p}{(}\PY{n}{W}\PY{o}{.}\PY{n}{shape}\PY{p}{[}\PY{l+m+mi}{0}\PY{p}{]}\PY{o}{/}\PY{o}{/}\PY{l+m+mi}{4}\PY{p}{)} \PY{p}{]}
         \PY{k}{for} \PY{n}{datatype}\PY{p}{,} \PY{n}{weights} \PY{o+ow}{in} \PY{n}{W\PYZus{}activities}\PY{o}{.}\PY{n}{items}\PY{p}{(}\PY{p}{)}\PY{p}{:}
             \PY{n}{plt}\PY{o}{.}\PY{n}{plot}\PY{p}{(}\PY{n}{time}\PY{p}{,} \PY{n}{weights}\PY{p}{,} \PY{n}{linewidth}\PY{o}{=}\PY{l+m+mi}{2}\PY{p}{,} \PY{n}{label}\PY{o}{=}\PY{n}{datatype}\PY{p}{)}
         \PY{n}{plt}\PY{o}{.}\PY{n}{plot}\PY{p}{(}\PY{p}{(}\PY{l+m+mi}{0}\PY{p}{,} \PY{n+nb}{max}\PY{p}{(}\PY{n}{time}\PY{p}{)}\PY{p}{)}\PY{p}{,} \PY{p}{(}\PY{l+m+mi}{0}\PY{p}{,} \PY{l+m+mi}{0}\PY{p}{)}\PY{p}{,} \PY{n}{linewidth}\PY{o}{=}\PY{l+m+mi}{1}\PY{p}{,} \PY{n}{color}\PY{o}{=}\PY{l+s+s1}{\PYZsq{}}\PY{l+s+s1}{k}\PY{l+s+s1}{\PYZsq{}}\PY{p}{,} \PY{n}{linestyle}\PY{o}{=}\PY{l+s+s1}{\PYZsq{}}\PY{l+s+s1}{\PYZhy{}\PYZhy{}}\PY{l+s+s1}{\PYZsq{}}\PY{p}{)}
         \PY{n}{plt}\PY{o}{.}\PY{n}{xlabel}\PY{p}{(}\PY{l+s+s1}{\PYZsq{}}\PY{l+s+s1}{time delay in minutes}\PY{l+s+s1}{\PYZsq{}}\PY{p}{)}
         \PY{n}{plt}\PY{o}{.}\PY{n}{ylabel}\PY{p}{(}\PY{l+s+s1}{\PYZsq{}}\PY{l+s+s1}{weight}\PY{l+s+s1}{\PYZsq{}}\PY{p}{)}
         \PY{n}{plt}\PY{o}{.}\PY{n}{xticks}\PY{p}{(}\PY{p}{[}\PY{l+m+mi}{0}\PY{p}{]} \PY{o}{+} \PY{n}{time}\PY{p}{)}
         \PY{n}{plt}\PY{o}{.}\PY{n}{xlim}\PY{p}{(}\PY{l+m+mi}{0}\PY{p}{,} \PY{n+nb}{max}\PY{p}{(}\PY{n}{time}\PY{p}{)}\PY{p}{)}
         \PY{n}{plt}\PY{o}{.}\PY{n}{legend}\PY{p}{(}\PY{p}{)}
         \PY{n}{plt}\PY{o}{.}\PY{n}{show}\PY{p}{(}\PY{p}{)}
\end{Verbatim}


    \begin{center}
    \adjustimage{max size={0.9\linewidth}{0.9\paperheight}}{MobileData_solutions_files/MobileData_solutions_22_0.png}
    \end{center}
    { \hspace*{\fill} \\}

    To interpret the figure, remember that W includes 32 numbers that are
constructued from the combination of 8 time delays and 4 activities. The
horizontal axis now represent the time delay and the different curves
represent the different activities. The large value of the movement
curve at 15 minutes shows us that the support for future movement is
dominated by current movement. If you are moving right now, then you are
likely to be moving in 15 minutes, and if you are not moving right now,
then you probably won't in 15 minutes. We see that calling and texting
likewise have a positive impact on the support for future movement.
However, social proximity in the time window leading up to the
prediction window actually has a negative impact on the support. This
kind of makes sense - one is less likely to initiate movement, when one
is enjoying the company of others. Generally, all any other impact has a
small positive impact on the support. The large positive impact of
movement at 120 minutes delay is probably due to the effect of classes
that last around 2 hours.


    % Add a bibliography block to the postdoc



    \end{document}
